\section{Descripción del Problema}
En Android, por defecto, el desarrollador no cuenta con mecanismos para
definir políticas de confidencialidad e integridad que regulen
el flujo de información de sus aplicaciones. Siendo complejo prevenir fugas de
información del usuario, puesto que, el desarrollador carece de herramientas que
le garanticen la ausencia de flujos indeseados.\newline
Precisamente, una de las principales preocupaciones de seguridad en aplicativos
Android, es la manipulación de información del usuario.
Así lo evidencia un
estudio reciente de seguridad en dispositivos móviles, publicado por
McAfee\cite{McAfeeReport}, este señala  que una importante cantidad de
aplicaciones Android invaden la privacidad del usuario, reuniendo información
detallada de su desplazamiento, acciones en el dispositivo, y su vida personal.
De este modo, 80\% reúnen información de la ubicación, 82\%
hacen seguimiento de alguna acción en el dispositivo, 57\%
registran la forma de uso del celular(mediante Wi-Fi o
mediante la red de telefonía), y 36\% conocen información de
las cuentas de usuario.\newline
Las motivaciones para este tipo de acciones varían acorde al tipo de
información, por ejemplo: monitorear información de ubicación para mostrar
publicidad no solicitada; seguir las acciones sobre el dispositivo, para conocer
qué aplicaciones son rentables de desarrollar, o para ayudar a aplicaciones
maliciosas a evadir defensas; acceder a información de cuentas del usuario con
fines delictivos; obtener información de contactos y calendario
del usuario, buscando modificar los datos; obtener información del
celular(número, estado, registro de MMS y SMS) para interceptar llamadas y
enviar mensajes sin consentimiento del usuario.\newline 
Con o sin autorización de acceso, existen motivaciones suficientes para que un
tercero desee manipular información del usuario.\newline
Adicionalmente, el informe señala que una aplicación invasiva no necesariamente
contiene malware, y que su finalidad no siempre implica fraude; de las
aplicaciones que más vulneran la privacidad del usuario, 35\% contienen
malware.\newline 
Si bien, aplicaciones invasivas no necesariamente implican malware y/o acciones
delictivas, el cuestionamiento de fondo es la forma y finalidad con que una
aplicación manipula la información del usuario, y qué garantías puede ofrecer el
desarrollador para que tal manipulación sea consentida.

La falta de control sobre los flujos de información de la aplicación puede
ocasionar fugas de información, generando problemas de seguridad tanto para
quien la implementa como para quien la usa.\newline
Como contramedida a este problema, la API de Android ofrece herramientas de
seguridad basadas en políticas de control de acceso, y el desarrollador puede
implementarlas en su aplicación. Sin embargo, estos mecanismos se centran en
regular el acceso de los usuarios del sistema a determinados recursos, y no en
verificar qué sucede con la información una vez es accedida. 

Para superar dicha carencia, diferentes trabajos de investigación han abordado
el problema de fuga de información en aplicaciones Android, tanto desde un enfoque
dinámico como desde un enfoque estático, la literatura existente al
respecto(TaintDroid\cite{TaintDroid}, Flow-Droid\cite{FlowDroid-Thesis},
DidFail\cite{DidFail}, DroidForce\cite{DroidForce}), indica que la mayoría de
propuestas hacen data-flow analysis mediante técnicas de análisis 
tainting, partiendo del bytecode
Enfocándose en la precisión y eficiencia del análisis para detectar fugas de
datos en aplicaciones de terceros ya implementadas. Por consiguiente,
la finalidad del análisis no es garantizar el cumplimiento de políticas de
confidencialidad e integridad desde la construcción del aplicativo.

Partir de tales propuestas para analizar aplicaciones propias y garantizar
políticas de confidencialidad e integridad desde su construcción, puede implicar
incompletitud en el análisis(under-tainting) y no detección de flujos
implícitos. Esto debido a que,
% aún cuando el
% desarrollador conoce la funcionalidad de su propio código, las optimizaciones
% realizadas por el compilador pueden adicionar complejidad al
% mismo\cite[pag.~43]{SecureProgramming}; 
por un lado, al realizar análisis tainting de forma dinámica, el marcado de
datos se propaga únicamente a través de caminos del programa actualmente
ejecutados. Así, si existen datos que son influenciados por los datos marcados,
pero no están dentro de los actuales caminos de ejecución, quedan sin la
propagación de la marca, dando lugar al problema de
undertainting\cite{dynamic-tainting}\cite{Bit-Level-Taint-Analysis}. Es
decir, se obtiene precisión en el análisis, pero se pierde completitud.\newline
Por el otro, aún cuando se hace análisis tainting de forma estática, y el
marcado de datos puede ser propagado para todos los caminos posibles de
ejecución  del programa, superando el inconveniente de under-tainting, la
detección de flujos implícitos es posible si, en la construcción de la
herramienta de análisis se propaga el marcado de datos para flujos
implícitos\cite{taint-analysis}. 
Sin embargo, las propuestas que basan su análisis en data-flow estático,
suelen restringir la propagación del marcado de datos a flujos explícitos,
ganando eficiencia en el análisis. Las propuestas mencionadas anteriormente, no
son ajenas a tal generalidad(DidFail\cite{DidFail}[page 33],
FlowDroid\cite{FlowDroid-Thesis}[page 30]).

Ahora bien, la falta de garantías en el cumplimiento de determinadas
políticas de seguridad en la aplicación que se implementa, puede superarse
usando control de flujo de información, Information Flow Control(IFC), puesto
que, con esta técnica la aplicación  es analizada estáticamente  para
identificar todos los posibles caminos que podrían tomar sus flujos de
información, garantizando que a tiempo de ejecución, la aplicación respeta
determinadas políticas de seguridad.

Finalmente, partiendo del contexto que se plantea, donde es el propio
desarrollador Android quien requiere evaluar políticas de seguridad en su
aplicación, para  garantizarle al usuario que la aplicación las cumple. Resulta
apropiado proveerle una herramienta de apoyo, mediante la cual analice el flujo
de información de la aplicación que implementa, y verifique el cumplimiento
de políticas de seguridad.
