\section{evaluación}
\label{sec:eval}
Esta sección inicia presentando el conjunto de evaluación del que se parte para
evaluar la herramienta de solución propuesta(Prototipo), más las métricas de
seguridad a medir.
Después compara los resultados de analizar el conjunto de evaluación con
el Prototipo, y otras dos herramientas: FlowDroid y JoDroid.

\subsection{Conjunto de evaluación y metricas}
Para la evaluación se parte de DroidBench versión
1.0\cite{DroidBenchBenchmarks}, DroidBench es un benchmark creado
específicamente para evaluar propiedades de seguridad en aplicaciones Android.
De este benchmark se toma un grupo de testcases que permiten evaluar la política
de seguridad establecida \ref{subsection:politica}, tal grupo está integrado por
20 testcases, de los cuales, 14 presentan fugas de información. En la tabla
\ref{tb:conjunto} se listan los casos de prueba.
\begin{table}[t]
\begin{center}
\small
\resizebox{6cm}{!}{
\begin{tabular}{|p{0,5cm}|p{5,2cm}|}
 	\hline
 	\multicolumn{2}{|c|}{\cellcolor{gray!30}Conjunto de Evaluación} \\
	\hline
	\textbf{Test} & \textbf{Nombre Testcase}\\
	\hline
	1 & AndroidSpecific\_DirectLeak1\\
	\hline
	2 & AndroidSpecific\_InactiveActivity\\
	\hline
	3 & AndroidSpecific\_LogNoLeak\\
	\hline
	4 & AndroidSpecific\_Obfuscation1\\
	\hline
	5 & AndroidSpecific\_PrivateDataLeak2\\
	\hline
	6 & ArraysAndLists\_ArrayAccess1\\
	\hline
	7 & ArraysAndLists\_ArrayAccess2\\
	 \hline
	8 & GeneralJava\_Exceptions1\\
	\hline
	9 &  GeneralJava\_Exceptions2\\
	\hline
	10 & GeneralJava\_Exceptions3\\
	\hline
	11 & GeneralJava\_Exceptions4\\
	\hline
	12 & GeneralJava\_Loop1\\
	\hline
	13 & GeneralJava\_Loop2\\
	\hline
	14 & GeneralJava\_UnreachableCode\\
	\hline
	15 & ImplicitFlows\_ImplicitFlow1\\
	\hline
	16 & ImplicitFlows\_ImplicitFlow2\\
	\hline
	17 & ImplicitFlows\_ImplicitFlow4\\
	\hline
	18 & Lifecycle\_ActivityLifecycle3\\
	\hline
	19 & Lifecycle\_BroadcastReceiverLifecycle1\\
	\hline
	20 & Lifecycle\_ServiceLifecycle1\\
	\hline
\end{tabular}
}
\end{center}
\caption{Conjunto de evaluación.\newline
Donde \textit{Item} identifica el testcase y \textit{Testcase} especifica el
nombre de la aplicación para el caso de prueba}
\label{tb:conjunto}
\end{table}
% % \vspace{-0.5em}
Este conjunto de prueba es analizado con FlowDroid, JoDroid y con el Prototipo. 
El Prototipo representa la herramienta de análisis propuesta, por tanto, con el
Prototipo se genera la versión anotada del aplicativo y se evalua la política
anotada.\newline
Los calificativos para los resultados del análisis son:  
True Positive(TP): cuando se reporta un leak que efectivamente existe. 
False Positive(FP): cuando se reporta un leak que no existe.  
True Negative(TN): cuando no se reporta leak y efectivamente no existe. 
False Negative(FN): cuando no se reporta un leak existente.\newline
Adicionalmente, con el comando \textit{time}\cite{time-man} de unix, se mide el
tiempo(desempeño) que tarda cada herramienta en ejecutar el análisis.\newline
En base a los resultados de análisis que reporta cada herramienta, se
calcula su respectiva Precisión y Recall.\newline
La \textbf{Precisión} hace referencia a los Casos Positivos esperados(correctos
e incorrectos: TP, FP), en contraste con la proporción de verdaderos Positivos(TP)
detectados\cite{Precision-Recall}. Una alta Precisión indica que la herramienta
reporta más correctos Positivos(TP) que incorrectos Positivos(FP).\newline 
El \textbf{Recall} indica la proporción de Casos Positivos detectados(TP),
frente a los Casos Positivos esperados como correctos\cite{Precision-Recall}. Un
alto Recall indica que la herramienta reporta más correctos Positivos(TP) que incorrectos
Negativos(FN), es decir, la herramienta deja pasar menos errores.\newline
Las fórmulas para calcular Precisión(p) y Recall(r), son:
\begin{equation}
\label{pre}
	p = TP/(TP +FP) 
\end{equation}
\begin{equation}
\label{rec}
	r = TP/(TP+FN)
\end{equation}
% \vspace{-0.5em}
\subsection{Comparación de Desempeño: FlowDroid, JoDroid y Prototipo}
En la tabla \ref{tb:resultados} se ilustran los resultados de analizar el
conjunto de pruebas con FlowDroid, JoDroid y el Prototipo. El campo \emph{test}
identifica los casos de prueba listados en la tabla \ref{tb:conjunto}.
De acuerdo al tiempo que toma cada herramienta para analizar cada caso de
prueba, tal como se registra en los campos tF, tJ y tP; el Prototipo presenta
mejor desempeño frente a FlowDroid y JoDroid. En el caso de FlowDroid, la
ejecución del análisis, tarda en promedio tarda 3,3 segundos más que el
Prototipo. En el caso de JoDroid, el tiempo de análisis es costoso, ya que su tiempo de
ejecución oscila entre 21 y 22 minutos.\newline
Como lo señalan los resultados de análisis, el Prototipo presenta mejor
desempeño, esto como resultado de analizar flujo de información mediante
lenguajes tipados de seguridad, más específicamente a través del lenguaje tipado
de seguridad Jif.
Dado que Jif recurre a técnicas de compilación y no requiere la generación de
grafos de dependencia, el análisis toma menos tiempo.
\begin{table}[t]
\begin{center}
\small
\resizebox{8,5cm}{!}{
\begin{tabular}{|p{0,5cm}|p{0,7cm}|p{0,5cm}|p{0,5cm}|p{0,5cm}|p{1cm}|p{1,6cm}|p{1cm}|}
	\hline
 	\multicolumn{8}{|c|}{\cellcolor{gray!30}Resultados de evaluación} \\
	\hline
	\textbf{Test} & \textbf{Leaks} & \textbf{F} &
	\textbf{J} & \textbf{P}&\textbf{ tF} & \textbf{ tJ} &\textbf{tP}\\
	\hline
	1 & 1 & TP & TP & TP &5.371s &
	22m11.991s &2.063s\\
	\hline
	2 & 0 & TN & FP & FP  &3.255s &
	22m25.617s &2.469s\\
	\hline
	3 & 0 & TN & TN & TN &5.505s & 21m6.548s &2.946s\\
	\hline
	4 & 1 & TP & TP & TP &6.734s &
	22m46.541 &2.706s\\
	\hline
	5 & 1 & TP & TP & TP & 6.144s &
	21m32.447s &2.644s\\
	\hline
	6 &  0 & FP & FP & FP & 4.708s & 22m01.926s &
	1.278s\\
	\hline
	7 &  0 & FP & FP & FP & 4.4s &
	22m11.023s &1.361s\\
	 \hline
	8 &  1 & TP & FN & TP &6.397s & 22m52.134s &2.755s\\
	\hline
	9 & 1 & TP & FN & TP &5.887s & 21m4.434s &1.980s\\
	\hline
	10 & 0 & FP & TN & FP &6.008s & 21m37.040s &2.032s\\
	\hline
	11 & 1 & TP & FN & TP &5.731s & 21m10.240s &2.313s\\
	\hline
	12 & 1 & TP & TP & TP &5.605s & 21m15.30s &2.800s\\
	\hline
	13 & 1 & TP & TP & TP &4.719s & 21m41.224s &1.361s\\
	\hline
	14 & 0 & TN & TN & FP &3.792s &
	22m84.138s &1.197s\\
	\hline
	15 & 1 & FN & TP & TP &4.853s &
	22m55.645s&1.331s\\
	\hline
	16 &  1 & FN & TP & TP &4.496s &
	22m32.231s&1.212s\\
	\hline
	17 & 1 & FN & TP & TP &4.375s &
	22m43.110s &1.224s\\
	\hline
	18 &   1 & TP & TP & TP &4.792s &
	22m54.651s &1.222s\\
	\hline
	19 &   1 & TP & TP & TP &4.456s &
	22m42.347s &1.061s\\
	\hline
	20 &  1 & TP & TP & TP &5.225s &
	22m92.722s&1.180s\\
	\hline
\end{tabular}
}
\end{center}
\caption{Resultados de evaluación para FlowDroid, JoDroid y Prototipo. Donde
\textit{test} indica el testcase que se evalúa; \textit{Leaks} especifica
el numero de fugas de información que presenta el testcase; \textit{F},
\textit{J} y \textit{P} muestran los resultados retornados por FlowDroid,
JoDroid y por el Prototipo; \textit{tF}, \textit{tJ} y \textit{tP}, señalan el tiempo
que toma el análisis para Flowdroid, JoDroid y para el Prototipo,
respectivamente.}
\label{tb:resultados}
\end{table}

\subsection{Precisión, Recall y Detección de Flujos Implícitos}
Partiendo de los resultados de evaluación de la tabla \ref{tb:resultados}, 
en la tabla \ref{tb:porcentajes} se resume el total de falsos positivos(FP),
verdaderos positivos(TP), verdaderos negativos(TN) y falsos negativos(FN),
reportados por cada herramienta. Con estos valores se calcula la Precisión y el
Recall, cuyos resultados son ilustrados en la tabla
\ref{tb:comparacion}.\newline 
Respecto a los resultados de la tabla \ref{tb:comparacion}, es posible decir: 
\emph{Precisión:} el análisis pesimista en que se basa el Prototipo, donde se
asume que todos los métodos implementados en la aplicación serán invocados, hace que los resultados
del análisis sean menos precisos, generando más falsos positivos(FP).\newline 
\emph{Recall:} Para la evaluación realizada, el análisis de flujo de información
mediante el lenguaje tipado de seguridad Jif, ofrece mejor Recall, frente al análisis
de flujo de datos en que se basa FlowDroid y el análisis de flujo de información
(mediante System Dependences Graphs) en que se basa JoDroid. En consecuencia,
para el experimento de evaluación, la técnica de análisis del Prototipo es
completa(completeness), puesto que, dentro de los leaks detectados están todos
los leaks que efectivamente existen.\newline 
\emph{Flujos Implícitos:} Una ventaja de las técnicas basadas en control de
flujo de información es que al analizar tanto flujos explícitos como flujos implícitos,
detectan la generación de leaks mediante flujos implícitos casi de forma
natural, contrario a lo que sucede en las técnicas de análisis de flujos de
datos, en estas, si la construcción del análisis no define casos de propagación
para el marcado de datos mediante flujos implícitos, el análisis carece de
criterios para la detección de fugas a través de los mismos.\newline
\begin{table}[t]
\begin{center}
\small
\resizebox{7cm}{!}{
\begin{tabular}{c|c|c|c|}
\cline{2-4}
& \cellcolor{gray!30}FlowDroid & \cellcolor{gray!30}JoDroid &
\cellcolor{gray!30}Prototipo \\
\cline{1-4}
\multicolumn{0}{ |c|  }{\multirow{0}{*}{FP} }  & 3 & 3 & 5\\ \cline{0-3}
\multicolumn{0}{ |c|  }{\multirow{0}{*}{FN} }  & 3 & 3 & 0\\ \cline{0-3}
\multicolumn{0}{ |c|  }{\multirow{0}{*}{TP} }  & 11 & 11 & 14\\\cline{0-3}
\multicolumn{0}{ |c|  }{\multirow{0}{*}{TN} }  & 3 & 3 &  1\\ \cline{0-3}
\end{tabular}
}
\end{center}
\caption{Resultados de precisión para FlowDroid y Prototipo. Resume el total de
FP, TP, TN y FN}
\label{tb:porcentajes}
\end{table}

\begin{table}[t]
\begin{center}
\small
\resizebox{8cm}{!}{
\begin{tabular}{c|c|c|c|}
\cline{2-4}
& \multicolumn{0}{ c|  }{\multirow{1}{*}{\cellcolor{gray!30} FlowDroid} } &
\cellcolor{gray!30}JoDroid & \cellcolor{gray!30}Prototipo \\
\cline{1-4}
\multicolumn{0}{ |c|  }{\multirow{0}{*}{Precisión} }  & 78,57\% & 78,57\% & 73,68\%
\\
\cline{0-3}
\multicolumn{0}{ |c|  }{\multirow{0}{*}{Recall} }  & 78,57 & 78,57\% &  100\%\\
\cline{0-3}
\multicolumn{0}{ |c|  }{\multirow{0}{*}{Flujos Implícitos} }  & No &
Si & Si\\
\cline{0-3}
\end{tabular}
}
\end{center}
\caption{Comparación entre FlowDroid, JoDroid y Prototipo. Ilustra los
porcentajes para Precisión, Recall, y la detección de leaks mediante
flujos implícitos.}
\label{tb:comparacion}
\end{table}

La tabla \ref{tb:res} resume las ventajas y desventajas entre el
Prototipo y las herramientas de comparación, FlowDroid y JoDroid. En esta se
indica si la herramienta presenta mayor(+) o menor(-): precisión, recall y costo
en desempeño. Si la herramienta permite(\checkmark) o no permite(X): detectar
fugas de información a través de flujos implícitos, detectar automátimente
sources y sinks, y anallizar varias aplicaciones(Análisis InterApp).\newline
\begin{table}[t]
\begin{center}
\small
\resizebox{8cm}{!}{
\begin{tabular}{|p{5,3cm}|p{1,1cm}|p{1,1cm}|p{1,1cm}|}
	\hline
	\cellcolor{gray!30} Item & \cellcolor{gray!30} Prototipo &
	\cellcolor{gray!30} FlowDroid & \cellcolor{gray!30} JoDroid\\
	\hline
	Precisión & - & + & +\\
	\hline
	Recall & + & - & -\\
	\hline
	Costo en desempeño & - & - & + \\
	\hline
	Detección de Flujos Implícitos & \checkmark & X & \checkmark \\
	\hline
	Detección automática de sources y sinks & X & \checkmark &  X\\
	\hline
	Soporte para análisis InterApp & X & \checkmark & X\\
	\hline
\end{tabular}}
\end{center}
\caption{Síntesis ventajas y desventajas del Prototipo frente a FlowDroid y
JoDroid,respectivamente.}
\label{tb:res}
\end{table}

Finalmente, los resultados de evaluación confirman las hipótesis iniciales
del presente trabajo, según las cuales se esperaba que: primero, al hacer análisis
de flujo de información mediante lenguajes tipados de seguridad, la herramienta estaría en
capacidad de detectar fugas de información tanto en flujos explícitos como en
flujos implícitos; segundo, los resultados del análisis serían más rápidos pero
menos precisos, reportando más falsos positivos que JoDroid y FlowDroid.\newline