\section{evaluación}
\subsection{Conjunto de evaluación y metricas}
Para la evaluación se parte de DroidBench versión
1.0\cite{DroidBenchBenchmarks}, DroidBench es un benchmark creado
específicamente para evaluar propiedades de seguridad en aplicaciones Android. 
De este benchmark se toma un grupo de testcases que permiten evaluar la política
de seguridad establecida \ref{subsection:politica}, tal grupo está integrado por
20 testcases, de los cuales, 14 presentan fugas de información. En la
tabla \ref{tb:conjunto} se listan los casos de prueba, cada caso de prueba es
itentificado mediante un número de item.

\begin{table}[t]
\begin{center}
\small
% \small\addtolength{\tabcolsep}{-3pt}
\resizebox{6cm}{!}{
\begin{tabular}{|p{0,5cm}|p{5,2cm}|}
 	\hline
 	\multicolumn{2}{|c|}{Conjunto de Evaluación} \\
	\hline
	\textbf{Item} & \textbf{Testcase}\\
	\hline
	1 & AndroidSpecific\_DirectLeak1\\
	\hline
	2 & AndroidSpecific\_InactiveActivity\\
	\hline
	3 & AndroidSpecific\_LogNoLeak\\
	\hline
	4 & AndroidSpecific\_Obfuscation1\\
	\hline
	5 & AndroidSpecific\_PrivateDataLeak2\\
	\hline
	6 & ArraysAndLists\_ArrayAccess1\\
	\hline
	7 & ArraysAndLists\_ArrayAccess2\\
	 \hline
	8 & GeneralJava\_Exceptions1\\
	\hline
	9 &  GeneralJava\_Exceptions2\\
	\hline
	10 & GeneralJava\_Exceptions3\\
	\hline
	11 & GeneralJava\_Exceptions4\\
	\hline
	12 & GeneralJava\_Loop1\\
	\hline
	13 & GeneralJava\_Loop2\\
	\hline
	14 & GeneralJava\_UnreachableCode\\
	\hline
	15 & ImplicitFlows\_ImplicitFlow1\\
	\hline
	16 & ImplicitFlows\_ImplicitFlow2\\
	\hline
	17 & ImplicitFlows\_ImplicitFlow4\\
	\hline
	18 & Lifecycle\_ActivityLifecycle3\\
	\hline
	19 & Lifecycle\_BroadcastReceiverLifecycle1\\
	\hline
	20 & Lifecycle\_ServiceLifecycle1\\
	\hline
\end{tabular}
}
\end{center}
\caption{Conjunto de evaluación.\newline
Donde \textit{Item} identifica el testcase y \textit{Testcase} especifica el
nombre de la aplicación para el caso de prueba}
\label{tb:conjunto}
\end{table}
Este conjunto de prueba es analizado con FlowDroid, JoDroid y con el
Prototipo(que representa la herramienta de solución propuesta). Los
calificativos para los resultados del análisis son:  
True Positive(TP): cuando se reporta un leak que efectivamente existe. 
False Positive(FP): cuando se reporta un leak que no existe.  
True Negative(TN): cuando no se reporta leak y efectivamente no existe. 
False Negative(FN): cuando no se reporta un leak existente.\newline
Adicionalmente, con el comando \textit{time}\cite{time-man} de unix, se mide el
tiempo(desempeño) que tarda cada herramienta en ejecutar el análisis.\newline
En base a los resultados de análisis que reporta cada herramienta, se
calcula su respectiva Precisión y Recall.\newline
La \textbf{Precisión} hace referencia a los Casos Positivos esperados(correctos
e incorrectos: TP, FP), en contraste con la proporción de verdaderos Positivos(TP)
detectados\cite{Precision-Recall}. Una alta Precisión indica que la herramienta
reporta más correctos Positivos(TP) que incorrectos Positivos(FP).\newline 
El \textbf{Recall} indica la proporción de Casos Positivos detectados(TP),
frente a los Casos Positivos esperados como correctos\cite{Precision-Recall}. Un
alto Recall indica que la herramienta reporta más correctos Positivos(TP) que incorrectos
Negativos(FN). Es decir, la herramienta deja pasar menos errores.\newline
Las fórmulas para calcular Precisión(p) y Recall(r), son:
\begin{equation}
\label{pre}
	p = TP/(TP +FP) 
\end{equation}
\begin{equation}
\label{rec}
	r = TP/(TP+FN)
\end{equation}

\subsection{Resultados de evaluación: FlowDroid, JoDroid y Prototipo}
La tabla \ref{tb:resultados} ilustra los resultados obtenidos al analizar el
conjunto de pruebas con cada una de las herramientas. Con el campo item se
identifican los respectivos casos de prueba, los cuales están
previamente definidos en la tabla \ref{tb:conjunto}.

\begin{table}[t!]
\begin{center}
\small
% \small\addtolength{\tabcolsep}{-3pt}
\resizebox{8,5cm}{!}{
\begin{tabular}{|p{0,5cm}|p{0,7cm}|p{0,5cm}|p{0,5cm}|p{0,5cm}|p{1cm}|p{1,6cm}|p{1cm}|}
	\hline
 	\multicolumn{8}{|c|}{Resultados de evaluación} \\
	\hline
	\textbf{Item} & \textbf{Leaks} & \textbf{F} &
	\textbf{J} & \textbf{P}&\textbf{ tF} & \textbf{ tJ} &\textbf{tP}\\
	\hline
	1 & 1 & TP & TP & TP &5.371s &
	22m11.991s &2.063s\\
	\hline
	2 & 0 & TN & FP & FP  &3.255s &
	22m25.617s &2.469s\\
	\hline
	3 & 0 & TN & TN & TN &5.505s & 21m6.548s &2.946s\\
	\hline
	4 & 1 & TP & TP & TP &6.734s &
	22m46.541 &2.706s\\
	\hline
	5 & 1 & TP & TP & TP & 6.144s &
	21m32.447s &2.644s\\
	\hline
	6 &  0 & FP & FP & FP & 4.708s & 22m01.926s &
	1.278s\\
	\hline
	7 &  0 & FP & FP & FP & 4.4s &
	22m11.023s &1.361s\\
	 \hline
	8 &  1 & TP & FN & TP &6.397s & 22m52.134s &2.755s\\
	\hline
	9 & 1 & TP & FN & TP &5.887s & 21m4.434s &1.980s\\
	\hline
	10 & 0 & FP & TN & FP &6.008s & 21m37.040s &2.032s\\
	\hline
	11 & 1 & TP & FN & TP &5.731s & 21m10.240s &2.313s\\
	\hline
	12 & 1 & TP & TP & TP &5.605s & 21m15.30s &2.800s\\
	\hline
	13 & 1 & TP & TP & TP &4.719s & 21m41.224s &1.361s\\
	\hline
	14 & 0 & TN & TN & FP &3.792s &
	22m84.138s &1.197s\\
	\hline
	15 & 1 & FN & TP & TP &4.853s &
	22m55.645s&1.331s\\
	\hline
	16 &  1 & FN & TP & TP &4.496s &
	22m32.231s&1.212s\\
	\hline
	17 & 1 & FN & TP & TP &4.375s &
	22m43.110s &1.224s\\
	\hline
	18 &   1 & TP & TP & TP &4.792s &
	22m54.651s &1.222s\\
	\hline
	19 &   1 & TP & TP & TP &4.456s &
	22m42.347s &1.061s\\
	\hline
	20 &  1 & TP & TP & TP &5.225s &
	22m92.722s&1.180s\\
	\hline
\end{tabular}
}
\end{center}
\caption{Resultados de evaluación para FlowDroid, JoDroid y Prototipo. Donde
\textit{Item} indica el testcase que se evalúa; \textit{Leaks} indica si el
testcase presenta fugas de información; \textit{F}, \textit{J} y  \textit{P}
muestran los resultados devueltos por FlowDroid, JoDroid y por el Prototipo;
\textit{tF}, \textit{tJ} y \textit{tP}, señalan el tiempo que toma el análisis
para Flowdroid, JoDroid y para el Prototipo, respectivamente.}
\label{tb:resultados}
\end{table}
Como dijdjdj
