\section{Contexto}
\subsection{Aplicaciones Android}
En esencia una aplicación Android es una aplicación Java con interfaces
descritas en sintaxis XML, cuya ejecución es activada por el framework la API
Android.\newline 
El framework de Aplicación Android ofrece diferentes funcionalidades de
operación del sistema, proporcionando información de los servicios ofrecidos por
el teléfono. Por ejemplo, provee información de la ubicación del
usuario.\newline 
Así pues, una aplicación Android obtiene del framework, clases e interfaces
necesarias para implementar sus funcionalidades.

Por otro lado, el SDK, Android Software Development Toolkit, permite compilar
la aplicación a una versión ejecutable por dispositivos Android, esto es, código
Dalvik bytecode(.dex). Adicionalmente, el SDK genera el APK, Android Application
Package, donde empaqueta todo el código de la aplicación, incluyendo el
bytecode. El APK es un archivo con extensión .apk, y es el que finalmente se
instala en el dispositivo para obtener las funcionalidades de la aplicación.

En lo que respecta a su estructura, 
una aplicación Android puede integrarse por uno o más de los siguientes
componentes: Activities, Services, Content Providers y Broadcast Receivers.\\
Las actividades representan acciones a ejecutar por el usuario, permiten que el
usuario se comunique con la aplicación.\\
Los servicios son componentes de aplicación que ejecutan tareas en background.\\
Los proveedores de contenido son componentes que permiten compartir datos entre
diferentes aplicaciones Android.\\
Los componentes Broadcast Receivers reciben mensajes enviados por el sistema o
por otras aplicaciones.

Los componentes que integran una aplicación son especificados en el archivo
manifiesto de la aplicación Manifest.xml\cite{App-Manifest}, donde
adicionalmente se declaran: tanto los permisos requeridos por la aplicación para
acceder a partes protegidas de la API\cite{Android-Permissions} e interactuar
con otras aplicaciones; como los permisos requeridos por aplicaciones
externas para interactuar con los componentes de la aplicación.

La comunicación entre varias aplicaciones Android(Comunicación interApp), tiene
lugar a través de intents\cite{App-Intent}, métodos proveídos por la API Android
para la activación de componentes tanto al interior de la aplicación como entre
aplicaciones externas. 

\subsection{Técnicas de análisis de código}
\subsubsection{Análisis estático y dinámico}
Las soluciones propuestas para detectar fuga de información en aplicaciones
Android, se enmarcan en el análisis estático o dinámico de la aplicación, en
algunos casos, se combinan ambos tipos.\newline 
En \textbf{análisis estático}\cite{Static-dynamic}, se estudia el código del
programa para inferir todos los posibles caminos de ejecución. Esto se logra
construyendo modelos de estado del programa, y determinando los estados posibles
a alcanzar por el programa.
No obstante, debido a que existen múltiples posibilidades de ejecución, se opta
por construir un modelo abstracto de los estados del programa. La consecuencia
de tener un modelo aproximado es pérdida de información y posibilidad de menor
precisión en el análisis.\newline 
Por otro lado, en \textbf{análisis dinámico} se ejecuta el programa y se analiza
su comportamiento, verificando el camino de ejecución que ha tomado el programa.
Esa exactitud en la ejecución que se verifica da precisión al análisis, porque
no es necesario construir un modelo aproximado de todos los posibles caminos de
ejecución.

\subsubsection{Técnicas utilizadas en análisis estático} 
Generalmente, para verificar el cumplimiento de políticas de seguridad mediante
análisis estático, se aplican técnicas de seguridad de tipado
(Typed-Inference/Security-Typed Analysis) y técnicas de flujo de
datos(Data/Control Flow Analysis)\cite{Information-Flow-Java}.\newline 
Con \textbf{técnicas Security-Typed} las propiedades de confidencialidad e
integridad son anotadas en el código, y verificadas a tiempo de compilación,
garantizando su cumplimiento a tiempo de ejecución.\newline 
Con \textbf{técnicas de flujo de control} y \textbf{técnicas de flujo de datos},
las políticas de seguridad son verificadas haciendo seguimiento al control de
flujo, o al flujo de datos, respectivamente. Estás técnicas suelen utilizar
grafos de Control de Flujo CFG(Control Flow Graph), Grafos de Flujo de Datos
DFG( Data Flow Graph) y Grafos de llamadas CG (Call Graphs).

\subsubsection{Security Typed Languages}
Las herramientas basadas en técnicas de análisis Security-Typed, involucran
conceptos como flujo de información, políticas de confidencialidad e integridad,
y chequeo de tipos.
\emph{Flujo de información}: el flujo de información describe el
comportamiento de un programa, desde la entrada de los datos hasta la salida de
los mismos.\newline 
\emph{Políticas de confidencialidad e integridad}: confidencialidad e integridad
son políticas de seguridad aplicables mediante control de flujo de información.
Mientras la confidencialidad busca prevenir que la información fluya hacia
destinos no apropiados, la integridad busca prevenir que la información provenga
de fuentes no apropiadas\cite{LanguageIFS-2013}.\newline
\emph{Chequeo de tipos}: al usar un lenguaje tipado de seguridad, las políticas
son definidas a través del lenguaje, porque son expresadas mediante anotaciones
en el código fuente del programa a verificar, y su evaluación se realiza
mediante chequeo de tipos.\newline 
El chequeo de tipos consiste en una técnica estática,
también utilizada para analizar flujo de información durante la compilación de
un programa, más específicamente en la etapa de análisis semántico, el
compilador identifica el tipo para cada expresión del programa y verifica que
corresponda al contexto de la expresión.
Bajo este principio de chequeo, lenguajes tipados de seguridad aplican
políticas de control de flujo, definiendo para cada expresión del programa un
tipo de seguridad(security type), de la forma:  tipo de dato y label de
seguridad(security label). Donde el label de seguridad regula el uso del dato,
acorde a su tipo.\newline 
El compilador realiza el chequeo de tipos, partiendo del conjunto de labels de
seguridad. Así, si el programa pasa el chequeo de tipos y compila correctamente,
se espera que cumpla con las políticas de control de flujo evaluadas.

\subsection{JIF(Java Information Flow)}
% \subsubsection{Sistema de anotaciones en Jif}
Jif es un lenguaje tipado de seguridad que extiende al lenguaje Java con labels
de seguridad, a través de los cuales se especifican restricciones de cómo
debería ser utilizada la información. Jif está compuesto por un compilador y un
sistema de anotaciones.\newline
El análisis de flujo de información de aplicativos Java mediante Jif, requiere
su implementación haciendo uso del sistema de anotaciones de Jif, de modo que se
especifiquen las políticas de seguridad a evaluar.
Tal implementación se basa en adicionar labels de seguridad a la definición
de métodos, variables, arrays, etc; los labels de seguridad no especificados son
generados automáticamente con labels por defecto.\newline
%http://www.cs.cornell.edu/jif/doc/jif-3.3.0/language.html#inference.
La verificación del cumplimiento de las políticas de seguridad, tiene lugar
durante la compilación del aplicativo, allí el compilador Jif aplica chequeo de
labels(label checking)\cite{jifRef},  verificando que los flujos de información
generados cumplen con las restricciones establecidas.

\subsubsection{DML(Decentralized Label Model)}
Jif basa su sistema de anotaciones en el modelo de etiquetas DLM, donde se
manejan tres elementos fundamentales: Principales, Políticas y
Etiquetas.\newline 
Principales: un principal es una entidad con autoridad para observar y cambiar
aspectos del sistema. Un programa pertenece a un principal, quien determina el
comportamiento que este debería tener. Jif cuenta con una serie de principals ya
definidos, por ejemplo, Alice, Bob, Chunck, etc, que pueden ser
utilizados al momento de anotar.\newline 
Políticas: mediante políticas de seguridad el dueño de la política, que es el
principal que la define, determina qué otros principals pueden leer o
influenciar la información. Así, una política puede ser de confidencialidad o de
integridad, y se especifican de la forma: \{owner: reader list\} u
\{owner: writer list\}.\newline 
Etiquetas: una etiqueta consiste en un conjunto de políticas de confidencialidad
e integridad. Las etiquetas se escriben en las expresiones del programa que se
anota(etiquetas de seguridad), esto es métodos, variables, arrays, etc.\newline 
En síntesis, las políticas de seguridad definen que Principales pueden leer o
modificar la información, y esas políticas se expresan mediante etiquetas.

\subsubsection{Chequeo de etiquetas}
Para hacer seguimiento al flujo de información de un programa, el compilador de
Jif asocia una etiqueta al program counter de cada punto del programa,
etiqueta del progam-counter(\underline{pc}). En cada punto del programa, el
(\underline{pc}) representa la información que podría conocerse tras la
ejecución de ese punto del programa.
El (\underline{pc}) es afectado por las etiquetas con que se define cada
sentencia y expresión del programa, por tanto este es considerado como el límite
superior(máxima información que podría conocerse) de las etiquetas que han
afectado el flujo de información para llegar a un determinado punto de ejecución.

\subsubsection{Anotación de variables y métodos en Jif}
\label{sssec:sintaxis}
dado que la sintaxis de anotación de Jif se basa en etiquetas de seguridad para
extender la sintaxis del lenguaje java, a continuación se ilustra la forma en
que normalmente se declaran variables y métodos en Java, y la forma en que se
realiza la respectiva extención con la sintaxis de Jif.\newline
\textbf{Definición de variables}\newline
% -Definición de variables: \newline 
En Java la sintaxis para definir una variable es:
\begin{lstlisting}[basicstyle=\scriptsize]
	modifier java-type varName
\end{lstlisting}
Extendiendo la sintaxis Java, en Jif las variables se definen de la forma:
\begin{lstlisting}[basicstyle=\scriptsize]
	modifier java-type {L} varName
\end{lstlisting}
Donde java-type especifica el tipo de dato Java que almacena la variable, \{L\}
la etiqueta de seguridad  para especificar quien es el principal dueño de la
variable, y varName, el respectivo nombre de la variable.\newline
\textbf{Definición de métodos}\newline
en Java la definición de un método tiene la siguiente sintaxis:
\begin{lstlisting}[basicstyle=\scriptsize]
modifier java-type methodName(java-type arg1,,, 
java-type argn){body method}
\end{lstlisting}
En Jif  es posible asociar una etiqueta de seguridad al tipo de dato retornado,
los argumentos que recibe y las excepciones declaradas. 
Adicionalmente, se declara un begin-label(BL) y un end-label(EL).\newline
Cuando en la definición del método no se específica una etiqueta de seguridad,
el compilador de Jif asume unas por defecto. 
La sintaxis es la siguiente:
\begin{lstlisting}[basicstyle=\scriptsize]
modifier java-type{RTL} methodName{BL}(java-type arg1{AL},,,
				java-type argn{AL}) :{EL}
\end{lstlisting}
Donde: \emph{java-type}, es el tipo de dato Java retornado por el
método.\newline 
\emph{RTL}, Return Type Label, indica la etiqueta de seguridad para el valor
retornado por el método.\newline 
\emph{BL}, Begin Label, representa el máximo nivel se seguridad del
\underline{pc} desde donde se invoca el método, de este modo,
la etiqueta del program counter desde donde se invoca el método debe ser
menor o igual de restrictivo que el BL con que se define el método. El BL
también asegura que el método sólo podrá actualizar partes del programa que
tengan igual BL. Con tales restricciones se evita la generación de flujos
implícitos, vía invocación del método.\newline
\emph{AL}, Argument Label, indica el máximo nivel de seguridad para los
argumentos con que se llama el método, así, las etiquetas de los argumentos con
que se invoca el método deben ser menor o igual de restrictivos que el AL con que ha
definido el método.\newline 
\emph{EL}, End Label, indica el \underline{pc} en el punto de terminación del
método, y representa el máximo nivel de información que puede conocerse tras la
finalización del método.
