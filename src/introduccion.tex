\section{Introducción}
En aplicativos Android, el manejo de la información del usuario, es una de las
principales preocupaciones de seguridad. Según un estudio reciente de seguridad
en dispositivos móviles, publicado por McAfee\cite{McAfeeReport}, una importante
cantidad de aplicaciones Android invaden la privacidad del usuario, reuniendo
información detallada de su desplazamiento, acciones en el dispositivo, y su
vida personal.\newline
Por otro lado, para controlar el acceso a información manipulada por sus
aplicaciones, el desarrollador cuenta con los mecanismos de seguridad proveídos
por la API de Android, sin embargo, al estar basados en políticas de control de
acceso, se limitan a verificar el uso de los recursos del sistema acorde a los
privilegios del usuario, lo que suceda con la información una vez sea accedida,
está fuera del alcance de este tipo de controles. Al no contar con herramientas
de análisis de flujo de información en aplicaciones Android, o al utilizar
librerías de terceros, para el desarrollador es difícil verificar
el cumplimiento de políticas de confidencialidad e integridad en la aplicación
próxima a liberar. Por consiguiente, el desarrollador no tiene cómo garantizarle
al usuario que la aplicación que le provee no presenta fugas de información.\newline
Ahora bien, aunque en el campo de aplicativos Android existen diferentes
propuestas para detectar fuga de información, en su mayoría  se enfocan en precisión y
eficiencia del análisis para detectar fugas de datos en aplicaciones de terceros
ya implementadas. Estas propuestas no abordan el problema del lado del
desarrollador, analizando flujos de información de la aplicación para verificar
el cumplimiento de políticas de seguridad.\newline
%  hacen falta propuestas que aborden
% el problema análizando flujo de información, 
% mediante técnicas de lenguajes tipados de seguridad(NO ESTOY SEGURA DE ESTA
% AFIRMACIÓN, ES DECIR QUE SÓLO CON SECURITY-TYPED ES POSIBLE DETECTAR FUGAS
% DE INFORMACIÓN),
% lo que se traduce en
% imposibilidad para detección de fugas mediante sentencias de control, por
% ejemplo, la no detección de flujos implicitos.\newline 
Ante esto, y con el fin de proveer una herramienta de apoyo al desarrollador, de
modo que pueda verificar el cumplimiento de políticas de seguridad desde la
construcción de sus aplicaciones, el presente trabajo aborda el problema de
fugas de información en aplicaciones Android, analizando flujos de información
de la aplicación, mediante técnicas de lenguajes tipados de seguridad.\newline
Así pues, en el presente trabajo se propone una herramienta para análisis de
flujo de información de aplicativos Android mediante el sistema de anotaciones
de Jif.\newline
Dado que Jif permite
anotar código Java pero no código Android, es decir, las anotaciones Jif son
válidas para clases del lenguaje Java estándar, no para clases específicas de la
API del framework Android, las cuales son indispensables para
implementar las funcionalidades de aplicativos Android; la principal
contribución de la propuesta consiste en: proveerle al desarrollador de
aplicativos Android una herramienta que permite definir y verificar políticas de
confidencialidad en sus aplicativos, con el sistema de anotaciones de
Jif.\newline

El resto del articulo está organizado de la siguiente manera: en la sección
\ref{sec:problema} se describe el problema de investigación, en la sección
\ref{sec:context} se presentan conceptos para contextualizar la solución del
problema, en la sección \ref{sec:propuesta} se plantea la propuesta de solución,
en la sección \ref{sec:trabajo} se presentan las diferencias entre la propuesta
planteada y los trabajos relacionados, en la sección \ref{sec:eval} se presentan
los resultados de evaluación de la propuesta, en la sección \ref{discusion} se
elaboran algunas discusiones y se plantea el trabajo futuro, finalmente en la
sección \ref{conclusiones} se presentan las conclusiones.


























