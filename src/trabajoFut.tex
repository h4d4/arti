\section{Discusión y Trabajo Futuro}
\subsection{Discusión}
\subsubsection{Límites de la solución propuesta}
Las limitaciones del diseño de solución en que se enfoca el presente trabajo,
corresponden a políticas de seguridad evaluables, tipos de aplicaciones
evaluables y limitaciones propias del lenguaje Jif.

\emph{Políticas de seguridad evaluables}\\
Se propone un esquema de anotación con niveles de seguridad alto y
bajo, que permite definir y evaluar políticas de confidencialidad en aplicativos
Android mediante el sistema de anotaciones de Jif.
Sin embargo, el esquema de anotación propuesto no permite evaluar políticas de
integridad ni aplicar mecanismos Downgrading, características ofrecidas por el
sistema de anotaciones de Jif.

\emph{Tipos de aplicaciones evaluables}\\
El diseño de solución en que se hace énfasis, para la
herramienta de análisis propuesta, no permite hacer análisis de flujo de
información vía interApp, es decir, no se hace seguimiento al flujo de
información durante el envío de mensajes intents para activar componentes de
aplicaciones externas.
%\textcolor{red}{POR QUE???}

\emph{Características del lenguaje Jif}\\
El compilador de Jif no soporta algunas características del lenguaje Java
estandar como: nested clases, initializer blocks, threads, etc; por tanto no es
posible analizar el flujo de información de aplicativos Android que requieran de
tales características del lenguaje Java, a menos que se encuentre sintaxis
equivalente para soportar tales características.\newline 

\subsubsection{Retos para el análisis}
El análisis de flujo de información en aplicativos Android mediante el sistema
de anotaciones de Jif, involucra una serie de retos como consecuencia de: las
diferencias entre una aplicación Android y una aplicación Java convencional; y
las limitaciones propias del lenguaje Jif.\newline 
Por un lado, Jif permite anotar código Java pero no código Android, es decir,
las anotaciones Jif son válidas para clases del lenguaje Java estándar, no para
clases específicas de la API del framework Android.\newline 
Ahora bien, aunque en esencia una aplicación Android es una aplicación Java con
interfaces descritas en sintaxis XML, una aplicación Android requiere de las
clases proveídas por el framework para la implementación de
funcionalidades.\newline 
Así, para analizar aplicativos Android con Jif, se
necesita anotar clases de la API Android mediante el sistema de anotaciones de
Jif, de modo que se posibilite el análisis de flujo de información.

Por otro lado, la anotación de las clases de la API Android(que en el fondo
están implementadas en lenguaje Java), están limitadas a las características del
lenguaje Java estándar reconocidas por el compilador de Jif.

Entre los retos que surgen están:\newline 
\emph{Clases de la API del framework y su soporte para Jif}\newline 
Anotar las clases de la API Android para el sistema de anotaciones de Jif no
es una tarea trivial debido a la cantidad(1) y a características específicas
del framework(2).

(1) Cantidad de clases a anotar: para analizar flujo de información en
aplicativos Android, lo ideal sería que todas las clases que conforman la API de
Android estuviesen anotadas a través del sistema de anotaciones de Jif, no
obstante, como la cantidad de clases que conforman la API es bastante amplía, en
la presente tesis se anotan únicamente las requeridas para verificar una
política de seguridad específica.
 
(2) Características específicas del framework: entre las funcionalidades del
framework está proveer las clases necesarias para interactuar con las interfaces
XML. No obstante, las clases involucradas para tales propósitos incluyen
operaciones específicas del framework no soportadas mediante el sistema de
anotaciones de Jif, puesto que, implican características del lenguaje Java
estándar que no son soportadas. Para las cuales, se adoptan mecanismos que
permitan analizar la información. Tales características incluyen:\newline 
- Casting entre tipos EditText y View: para procesar información proveniente de
campos de interfaces XML, se requiere hacer casting entre los tipos de datos representados por las clases EditText y
View. Sin embargo, este tipo de casting no es soportado por el compilador
Jif.\newline 
- Clase Nested R: para referenciar los
recursos(strings, estilos, widgets, layouts, e interfaces xml) necesarios para
una aplicación, el framework genera la clase R.java. Ahora bien, el
inconveniente para anotar ese tipo de clases con Jif, es que los identificadores
son descritos mediante clases nested, y esa característica del lenguaje Java
está dentro de las limitaciones del lenguaje Jif.\newline 
- Sobreescritura de componentes: para la implementación de aplicativos en Android es necesaria la
sobreescritura de componentes específicos de la API Android, componentes como
(activities, content providers, receivers, services), y cómo para indicar la
sobreescritura de los respectivos métodos de un componente se requiere el
Statement @Override cuya clase no es soportada por el sistema de anotaciones de
Jif.

\emph{Características específicas del lenguaje Java estándar}\newline
Otro reto importante para analizar aplicativos Android a través del sistema de
anotación de Jif, es que la sintaxis del lenguaje java en la implementación del
aplicativo se restringe a la sintaxis soportada por Jif, por ejemplo: Jif no
soporta el enhanced for loop\footnote{El enhanced for es una versión mejorada
del for, utilizado para simplificar la iteración en arrays y colecciones.}.
En consecuencia, habría que extender el compilador de Jif para que efectivamente reconozca características
adicionales del lenguaje Java.

En síntesis, los retos emergentes implican extensiones al sistema de
anotaciones de Jif, tanto para soportar clases específicas del framework de
Android, como para soportar características del lenguaje Java estándar no
soportadas por Jif.\newline
Adicionalmente, para las características del framework Android que
definitivamente no se puedan soportar mediante el sistema de anotaciones de Jif,
es necesario adoptar mecanismos que permitan analizar el flujo de información en
las aplicaciones que implementen tales características.\newline

\subsection{Trabajo Futuro}
Exceptuando las características del lenguaje Java que no son soportadas por el
compilador de Jif(nested clases, initializer blocks, threads, etc), se podría
ampliar el setup de Jif para clases de la API de Android, de modo que se
anote con el sistema de anotaciones de Jif la mayor cantidad de clases
posibles de la API.
Esto permitiría hacer análisis de flujo de información a aplicaciones
Android más robustas. Por ejemplo, si se anotan todas las clases
correspondientes para el manejo de intents(mensajes utilizados principalmente
para comunicar diferentes aplicaciones), sería posible incluir en el análisis de
flujo de información, el análisis vía interApp. 

El esquema de anotación propuesto podría ser
extendido para definir y analizar políticas de integridad y mecanismos
adicionales como declasificación y endorsement, verificables mediante el modelo
de anotaciones de Jif. De este modo, el desarrollador también podría garantizar
el cumplimiento de políticas de integridad, y contaría con mecanismos que le
permitan flexibilizar la definición de las políticas tanto de confidencialidad
como de integridad.
