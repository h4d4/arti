\section{Trabajos Relacionados}
\label{sec:trabajo}

Esta sección presenta otras propuestas de solución existentes para el análisis
de seguridad en aplicativos Android. Adicionalmente, indica el elemento
diferenciador de la propuesta planteada(\ref{sec:propuesta}) frente a estas.

\subsection{JOANA}
\label{JOANA-Tool}
JOANA (Java Object-sensitive ANAlysis)- Information Flow Control Framework for
Java\cite{JOANA}. Verifica si una aplicación java contiene fugas de
información, mediante análisis estático de flujos de información. El análisis parte  de anotaciones en
el código fuente de la aplicación. JOANA utiliza técnicas de análisis de flujo de
datos y técnicas de análisis de control de flujo. El frontend de la herramienta
está basado en el framework de análisis de programas WALA\cite{wala}, a partir
del cual obtiene la representación intermedia del código Java en forma SSA(Static
Single Assignement), lo que permite obtener información dinámica del programa.
Por otro lado, utiliza Grafos de Dependencia, System Dependence Graphs(SDG),
para detectar dependencias entre las sentencias del programa, es decir,
si existen caminos entre sentencias etiquetadas con nivel de seguridad
alto y sentencias con nivel de seguridad bajo. Para esta etapa del análisis
recurre a técnicas de slicing y chopping, reduciendo la cantidad de caminos
posibles sólo a los válidos. Así obtiene como resultado, una mayor precisión y
reducción de falsas alarmas en el análisis.

Aunque JOANA provee sencillez a la hora de anotar el código a analizar, pues
sólo es necesario anotar inputs y outputs del programa, porque la herramienta se
encarga de propagar las anotaciones en el resto del programa; carece de
características adicionales ofrecidas por sistemas de tipado de seguridad, por
ejemplo, el mecanismo downgrading facilitado por JIF.

Si bien, al igual que JOANA, la herramienta propuesta a través del presente
trabajo, aplica análisis de control de flujo de información, esta última busca
analizar aplicaciones implementadas en código Android, aprovechando las ventajas
del sistema de anotaciones de JIF. Proporcionando una herramienta de apoyo al
desarrollador de aplicaciones Android, ya que por el momento, JOANA sólo analiza
aplicaciones en JAVA.

\subsection{JoDroid}
\label{subsec:jod}
JoDroid\cite{JoDroid-Paper} es una extensión a la herramienta de análisis JOANA
para soportar análisis de aplicaciones Android.\newline 
El análisis de JOANA está basado en Program Dependence Graphs(PDG) y técnicas
slicing. Con PDGs obtiene una representación del programa que
analiza, donde los nodos representan statements y expresiones; y las aristas
modelan las dependencias sintácticas entre los statements y expresiones:
dependencias de datos y dependencias de control, por tanto el grafo está en
capacidad de modelar, tanto flujos explícitos como flujos implícitos.\newline
Con técnicas slicing provee sensibilidad al contexto, puesto que el PDG se
construye de manera tal que al hacer el backwards slice de un determinado nodo,
se obtiene cada nodo que es alcanzable por caminos del grafo que conservan
llamadas al contexto.\newline
El PDG es generado mediante el Front-end de WALA, framework que analiza bytecode
de Java. Así, los ajustes hechos a JOANA adaptan parte del Front-end de WALA
para generar el PDG de aplicaciones Android.\newline
JoDroid detecta tanto flujos explícitos como flujos implícitos.

\subsection{FlowDroid}
\label{FlowDroid-Tool}
FlowDroid es una herramienta para análisis estático de flujo de datos en
Aplicaciones Android. También permite el análisis de aplicaciones Java.\newline
Esta herramienta utiliza un tipo especial de análisis de flujo de datos:
análisis tainting, que hace seguimiento al flujo de datos entre un conjunto de
sources y un conjunto de sinks. Define tales conjuntos a partir de
SuSi[\ref{sec:susi}], un clasificador automático de sources y sinks para la API
de Android.\newline 
FlowDroid provee un alto recall y precisión\cite{FlowDroid-Thesis} en el
análisis. El recall, mediante un fiel modelamiento del ciclo de vida de una
aplicación Android; la precisión, incluyendo elementos de análisis como:
context-, flow-, field- y object-sensitive. Para proveer sensibilidad al flujo y
al contexto, recurre a grafos de llamada; y con grafos que modelan todos los
procedimientos del programa(inter-procedural control-flow graph), analiza el
flujo de datos entre procedimientos, proporcionando field- y object-sensitive.\newline
Los autores de esta propuesta, alcanzan precisión en la construcción del grafo
de llamadas extendiendo Soot\cite{Soot}, un framework que genera código
intermedio para código Java y código ejecutable Android(dex). Adicionalmente,
con el framework Heros\cite{heros}, incluyen llamadas multihilos en el análisis
de flujo de datos entre procedimientos.

Entre las limitaciones de FlowDroid está el over-tainting y la no detección
de flujos implícitos. Por tanto, la herramienta no distingue elementos marcados
ni dentro de arrays, ni dentro de collections, si se inserta un elemento marcado
dentro de alguna de estas estructuras, inmediatamente se marca el resto de
elementos. La herramienta tampoco identifica flujos implícitos,    
% causados por dependencias entre control de flujo.\newline
puesto que, según los resultados de evaluación de
DroidBench\cite{DroidBenchBenchmarks}, su benchmark; cuando Flowdroid analiza el
conjunto de aplicaciones de prueba para la identificación de flujos implícitos, no
detecta fuga de datos, generando falsos negativos en la detección de flujos
implícitos\cite[pags 32-36]{FlowDroid-Thesis}.

Aún cuando el problema a atacar es el mismo: fuga de información, la propuesta
que se expone a través del presente trabajo difiere en el enfoque de análisis de
FlowDroid, mientras FlowDroid se concentra en detectar si la aplicación de un
tercero presenta fugas de información, la herramienta planteada aborda el
análisis del lado del desarrollador de la aplicación, apoyándolo en
la verificación del cumplimiento de políticas de seguridad. Así, resulta más
conveniente guiar el análisis mediante control de flujo de información, ya que
se previene fuga por datos no marcados para el análisis(under-tainting) y por
la no detección de flujos implícitos, siendo posible garantizar el cumplimiento
de políticas de seguridad.
 
\subsection{TaintDroid, Dinamic Taint Tracking, para la detección de fugas de
Información}
\label{TaintDroid-Tool}
A diferencia de las propuestas expuestas anteriormente, caracterizadas
por ejecutar el análisis de manera estática, TaintDroid es una herramienta de
análisis dinámico. Está herramienta extiende la plataforma de dispositivos
celulares Android, con el fin de verificar el uso dado por aplicaciones de
terceros a datos sensibles del usuario. El análisis aplica técnicas de análisis
tainting, marcando automáticamente como sources, datos provenientes de fuentes
consideradas privadas y/o sensibles; y como sinks, canales para la salida de
datos de la aplicación, como por ejemplo internet.
Cada vez que un dato marcado como source sale de la aplicación, se genera un log.\newline 
Para reducir sobrecarga en el dispositivo, pues el análisis es ejecutado a nivel
de instrucciones, instrumentan la máquina virtual de Android con marcas de
propagación a nivel de: variables, métodos, mensajes y archivos. Las marcas de
variable hacen seguimiento a datos dentro de aplicaciones consideras no
confiables. Las marcas de mensaje siguen mensajes entre aplicaciones. Debido a
que TaintDroid no hace seguimiento a la ejecución de código nativo, utiliza las
marcas de métodos para hacer seguimiento a lo retornado luego de invocar métodos
de librerías nativas. Las marcas de archivo son utilizadas para verificar la
persistencia de los datos, acorde a las políticas de seguridad.\newline
Otra medida para reducir sobrecarga en la ejecución del análisis, consiste en
omitir flujos de control, generando no detección de flujos
implícitos\cite[pag 12]{TaintDroid}.\newline Si bien, TaintDroid supera el inconveniente de sobrecarga en la ejecución del
análisis, un inconveniente característico en análisis dinámico, está limitado
para detectar fuga de datos mediante flujos implícitos, puesto que se
enfoca en hacer seguimiento a flujos de datos directos(flujos
explícitos).

Al ser una herramienta de análisis dinámico, TaintDroid sólo detecta fugas de
información correspondiente a las ejecuciones presentadas por el programa, y
para la finalidad de su análisis: informar al usuario de posibles fugas de
información, se puede decir que es adecuado. No obstante, para los propósitos de
la propuesta planteada a través del presente trabajo, con la que se pretende
brindar una herramienta de análisis para que el desarrollador verifique el
cumplimiento de políticas de seguridad en la aplicación que implementa, no
resulta viable aplicar análisis dinámico, ni técnicas de análisis tainting para
hacer seguimiento a flujos de datos.
