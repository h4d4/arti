\section{Conclusiones}
\label{conclusiones}
El presente trabajo de investigación ha abordado la problemática enfrentada por
el desarrollador de aplicaciones Android, a la hora de definir políticas de
seguridad que regulen el flujo de información de sus aplicaciones. Puesto que,
aún cuando la API Android ofrece mecanismos de control de acceso y el
desarrollador puede implementarlos en sus aplicaciones, estos se centran en
regular el acceso de los usuarios a determinados recursos del sistema, y no en
verificar qué sucede con la información una vez es accedida.

Buscando contribuir con la solución de tal problemática, se propone una
herramienta de análisis estático basada en el sistema de anotaciones de Jif, que
permita analizar flujo de información en aplicativos Android. El diseño ideal
para la propuesta de solución, implica extender el setup de Jif para la API de
Android e incluir un clasificador de sources y sinks. Sin embargo, para efectos
de la presente tesis se limita el setup y el conjunto de sources y sinks, acorde
a una política de seguridad específica.

El diseño de solución en que se hace énfasis para la herramienta de análisis
estático, es evaluado y los resultados obtenidos son comparados frente a otras
herramientas de análisis estático: FlowDroid y Jodroid. Partiendo de los tipos
de análisis y técnicas evaluadas, de sus ventajas y desventajas, se
puede concluir:\newline 
- Con el sistema de anotaciones de Jif es posible proveer una
herramienta de apoyo al desarrollador de aplicaciones Android, de tal manera que evalúe el
cumplimiento de políticas de seguridad desde la construcción de sus aplicativos.\\
No obstante, el desarrollador debe adquirir un conocimiento previo de la
implementación de aplicativos en Jif.

- Al estar basado en análisis de flujo de información, Jif analiza tanto flujos
explícitos como flujos implícitos, ofreciendo la ventaja de detección de fugas
de información a través de flujos implícitos, sin requerirse trabajo adicional
para que el análisis incluya tales flujos. Contrario a lo que sucede con las
técnicas de análisis tainting, pues para incluir flujos implícitos en el
análisis, se requiere especificar casos que propaguen el marcado de datos a
través de dichos flujos.

- Al tratarse de análisis de flujo de información mediante lenguajes tipados de
seguridad, se obtienen las ventajas de desempeño y completitud en el análisis,
pero al mismo tiempo se obtiene como desventaja una baja precisión.\newline 
Las ventajas de desempeño obedecen a que el análisis se centra en técnicas de
chequeo de tipos(label checking), que al corresponder a etapas de compilación
dan como resultado bajo costo en desempeño.\newline
La completitud en el análisis se obtiene  haciendo seguimiento al flujo de
información de inicio a fin del aplicativo\cite{LanguageIFS-2013}, incluyendo
tanto flujos implícitos como flujos explícitos, generando así, un menor 
reporte de falsos negativos.\newline 
La baja precisión en el análisis obedece a
un enfoque de análisis pesimista, en el que, al incluir todas las posibles ramas de ejecución, se generan más
falsos positivos.

- Además de las ventajas y desventajas, el análisis de flujo de información en
aplicativos Android por medio del sistema de anotaciones de Jif,
comprende varios retos que implican extensiones al sistema de anotaciones de
Jif, tanto para soportar clases específicas del framework de Android, como para
soportar características del lenguaje Java estándar no soportadas por Jif.\newline 
Adicionalmente, para las características del framework Android que
definitivamente no se puedan soportar mediante el sistema de anotaciones de Jif,
es necesario adoptar mecanismos que permitan analizar el flujo de información en
las aplicaciones que las requieran.

- Como conclusión final, es pertinente resaltar que el presente trabajo de
investigación explora el análisis de flujo de información de aplicativos Android
mediante el sistema de anotaciones de Jif, y que,  aunque quedan bastantes retos
por superar, el principal aporte es que se posibilita el análisis de flujo de
información en aplicativos Android, mediante el sistema de anotaciones de Jif,
brindando una herramienta de apoyo al desarrollador para que verifique el
cumplimiento de determinadas políticas de seguridad, desde la construcción del
aplicativo.
